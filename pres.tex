\documentclass[aspectratio=169]{beamer}
\setbeamertemplate{navigation symbols}{}

%Compile with !pdflatex % and !biber slides
\usepackage[backend=biber,style=chem-acs]{biblatex}
\addbibresource{pres.bib}
\AtBeginEnvironment{frame}{\setcounter{footnote}{0}}
\setbeamertemplate{footline}[frame number]

\usepackage{graphicx}
\usepackage{amsmath}
\usepackage{braket}
\usepackage{verbatim}

\renewcommand\multicitedelim{\addsemicolon\space}
\newcommand\blfootnote[1]{%
  \begingroup
  \renewcommand\thefootnote{}\footnote{#1}%
  \addtocounter{footnote}{-1}%
  \endgroup
}

\title{Introduction to a Couple of Algorithms}
\author{Harper Grimsley}
%\institute{Mayhall Group, Virginia Tech}
\date{TBD}

\begin{document}

\frame{\titlepage}

\begin{frame}
	\frametitle{Why Quantum Chemistry is Hard}
	\begin{itemize}[<+->]
		\item Schr{\"o}dinger says to solve:
			\begin{equation}\nonumber
			\mathcal{H}\ket{\Psi} = E\ket{\Psi}
			\end{equation}
		\item $\ket{\Psi}$ includes exponentially many configurations- exponential memory to store $\ket{\Psi}$
		\item Dimension of $\mathcal{H}$ scales exponentially- diagonalizing it requires exponential time
	\end{itemize}
\end{frame}

\begin{frame}
	\frametitle{Quantum Computers and Storage}
	\begin{itemize}[<+->]
		\item Quantum registers of $N$ qubits exist in linear combinations of $2^N$ configurations
		\item Jordan-Wigner maps all configurations of $N$ spinorbitals to $N$ qubits
		\item E.g. H$_2$ in minimal basis:
			\begin{equation}\nonumber
				c_0\ket{\phi_0} + c_{0\bar{0}}^{1\bar{1}}\ket{\phi_{0\bar{0}}^{1\bar{1}}}\rightarrow c_{\uparrow\uparrow\downarrow\downarrow}\ket{\uparrow\uparrow\downarrow\downarrow}+c_{\downarrow\downarrow\uparrow\uparrow}\ket{\downarrow\downarrow\uparrow\uparrow}
			\end{equation}
		\item This is just a 4-electron spin state: one for each spinorbital
	        \item Number of qubits scales linearly with system size!
	\end{itemize}
\end{frame}

\begin{frame}
	\frametitle{Quantum Phase Estimation}
	\begin{itemize}[<+->]
		\item We can (maybe) get $E$ in polynomial time with QPE\footfullcite{abrams_quantum_1999}
        \end{itemize}
\end{frame}

\begin{frame}
	\frametitle{Block Encoding}
\end{frame}

\end{document}
